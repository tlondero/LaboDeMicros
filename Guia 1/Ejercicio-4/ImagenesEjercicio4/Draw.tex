\input{Header-Draw.tex}
\input{fillarea.tex}

\begin{document}

%%%%%%%%%%%%%%%%%%%%%%%%%%%%%%%%%%%%%%%%%%%%%%%%%%%%%%%%%%%%%%%%%%%%%
%							COMANDOS								%
%%%%%%%%%%%%%%%%%%%%%%%%%%%%%%%%%%%%%%%%%%%%%%%%%%%%%%%%%%%%%%%%%%%%%

%Eje vertical que compara los niveles 	(CONTINUO)
\newcommand{\barra}[8] % #1 = Xpos #2 = name #3 = VOL #4 = VIL #5 = Vt #6 = VIH #7 = VOH #8 = VCC
{
	\node [below] at (#1,-0.5) {#2};
	
	%Barras anchas
	\draw[-][draw=red, line width=2mm] (#1,#7) -- (#1,#8);
	\draw[-][draw=red, line width=2mm] (#1,0) -- (#1,#3);
	
	\draw[-][draw=gray, line width=2mm] (#1,#6) -- (#1,#7);
	\draw[-][draw=gray, line width=2mm] (#1,#3) -- (#1,#4);	
	
	\draw[-][draw=lightgray, line width=2mm] (#1,#4) -- (#1,#6);	
	
	%Eje vertical limites
	\draw[-][draw=black, very thick] (#1,0) -- (#1,#8);
	\draw[-][draw=black, very thick] (-0.2 + #1,0) -- (0.2 + #1,0);
	\draw[-][draw=black, very thick] (-0.2 + #1,#8) -- (0.2 + #1,#8);
	
	\node [left] at (-0.2 + #1,#8 + 0.2) {#8 V};
	\node [right] at (0.2 + #1,#8 + 0.2) {$V_{CC}$};	
	\node [left] at (-0.2 + #1,-0.2) {0 V};
	\node [right] at (0.2 + #1,-0.2) {$GND$};
	
	%Punto medio
	\draw[-][draw=black, very thick] (-0.1 + #1,#5) -- (0.1 + #1,#5);
	\node [left] at (-0.1 + #1,#5) {#5 V};
	\node [right] at (0.1 + #1,#5) {$V_t$};
	
	%VOH
	\draw[-][draw=black, very thick] (-0.1 + #1,#7) -- (0.1 + #1,#7);
	\node [left, color = gray] at (-0.1 + #1,#7) {#7 V};
	\node [right, color = gray] at (0.1 + #1,#7) {$V_{OH}$};
	
	%VIL
	\draw[-][draw=black, very thick] (-0.1 + #1,#4) -- (0.1 + #1,#4);
	\node [left, color = gray] at (-0.1 + #1,#4) {#4 V};
	\node [right, color = gray] at (0.1 + #1,#4) {$V_{IL}$};
	
	%VIH
	\draw[-][draw=black, very thick] (-0.1 + #1,#6) -- (0.1 + #1,#6);
	\node [left, color = red] at (-0.1 + #1,#6) {#6 V};
	\node [right, color = red] at (0.1 + #1,#6) {$V_{IH}$};
	
	%VOL
	\draw[-][draw=black, very thick] (-0.1 + #1,#3) -- (0.1 + #1,#3);
	\node [left, color = red] at (-0.1 + #1,#3) {#3 V};
	\node [right, color = red] at (0.1 + #1,#3) {$V_{OL}$};	
	
}

%Eje vertical que compara los niveles	(DISCONTINUO)
\newcommand{\barraDis}[8] % #1 = Xpos #2 = name #3 = VOLL #4 = VOLH #5 = VILL #6 = VILH #7 = VTL #8 = VTH #9 = VIHL #10 = VIHH #
{
	\node [below] at (#1,-0.5) {#2};
	
	%Barras anchas
	\draw[-][draw=red, line width=2mm] (#1,#7) -- (#1,#8);
	\draw[-][draw=red, line width=2mm] (#1,0) -- (#1,#3);
	
	\draw[-][draw=gray, line width=2mm] (#1,#6) -- (#1,#7);
	\draw[-][draw=gray, line width=2mm] (#1,#3) -- (#1,#4);	
	
	\draw[-][draw=lightgray, line width=2mm] (#1,#4) -- (#1,#6);	
	
	%Eje vertical limites
	\draw[-][draw=black, very thick] (#1,0) -- (#1,#8);
	\draw[-][draw=black, very thick] (-0.2 + #1,0) -- (0.2 + #1,0);
	\draw[-][draw=black, very thick] (-0.2 + #1,#8) -- (0.2 + #1,#8);
	
	\node [left] at (-0.2 + #1,#8 + 0.2) {#8 V};
	\node [right] at (0.2 + #1,#8 + 0.2) {$V_{CC}$};	
	\node [left] at (-0.2 + #1,-0.2) {0 V};
	\node [right] at (0.2 + #1,-0.2) {$GND$};
	
	%Punto medio
	\draw[-][draw=black, very thick] (-0.1 + #1,#5) -- (0.1 + #1,#5);
	\node [left] at (-0.1 + #1,#5) {#5 V};
	\node [right] at (0.1 + #1,#5) {$V_t$};
	
	%VOH
	\draw[-][draw=black, very thick] (-0.1 + #1,#7) -- (0.1 + #1,#7);
	\node [left, color = gray] at (-0.1 + #1,#7) {#7 V};
	\node [right, color = gray] at (0.1 + #1,#7) {$V_{OH}$};
	
	%VIL
	\draw[-][draw=black, very thick] (-0.1 + #1,#4) -- (0.1 + #1,#4);
	\node [left, color = gray] at (-0.1 + #1,#4) {#4 V};
	\node [right, color = gray] at (0.1 + #1,#4) {$V_{IL}$};
	
	%VIH
	\draw[-][draw=black, very thick] (-0.1 + #1,#6) -- (0.1 + #1,#6);
	\node [left, color = red] at (-0.1 + #1,#6) {#6 V};
	\node [right, color = red] at (0.1 + #1,#6) {$V_{IH}$};
	
	%VOL
	\draw[-][draw=black, very thick] (-0.1 + #1,#3) -- (0.1 + #1,#3);
	\node [left, color = red] at (-0.1 + #1,#3) {#3 V};
	\node [right, color = red] at (0.1 + #1,#3) {$V_{OL}$};	
	
}

%%%%%%%%%%%%%%%%%%%%%%%%%%%%%%%%%%%%%%%%%%%%%%%%%%%%%%%%%%%%%%%%%%%%%
%								DIBUJOS								%
%%%%%%%%%%%%%%%%%%%%%%%%%%%%%%%%%%%%%%%%%%%%%%%%%%%%%%%%%%%%%%%%%%%%%

%DIBUJO 1
\begin{page}
\begin{tikzpicture}[yscale=1]

	%#1 = Xpos #2 = name #3 = VOL #4 = VIL #5 = Vt #6 = VIH #7 = VOH #8 = VCC
	
	\barra{0}{5 V TTL}{0.4}{0.8}{1.55}{2}{2.4}{5}
	
	\barra{5}{3.3 V LVTTL}{0.4}{0.8}{1.55}{2}{2.4}{3.3}	

\end{tikzpicture}
\end{page}

%DIBUJO 2
\begin{page}
\begin{circuitikz}[american voltages]	
		
	\draw
		(0,0) node[nigfete,solderdot, rotate = -90, label=below:2N7000](fet){}
		
		(fet.S) -| node[pos=.5, circ](r1){} (-2,-1) node[ocirc, label=below:$3.3 \ v$ Signal](){}
		(fet.D) -| node[pos=.5, circ](r2){} (2,-1) node[ocirc, label=below:$5 \ v$ Signal](){}
		
		(r1.center) to[R, label=$10 \ K$] ++ (0,2) node[circ](aux1){} -| (fet.G)
		(aux1.center) -- ++ (-1,0) node[ocirc, label=above:$3.3 \ V$](){}
		
		(r2.center) to[R, label=$10 \ K$] ++ (0,2) -- ++ (1,0) node[ocirc, label=above:$5 \ V$](){}		
	;

\end{circuitikz}
\end{page}

%DIBUJO 3
\begin{page}
\begin{tikzpicture}[yscale=1]

	%#1 = Xpos #2 = name #3 = VOL #4 = VIL #5 = Vt #6 = VIH #7 = VOH #8 = VCC
	
	\barra{0}{CMOS}{0.2}{1.3}{2.5}{3.7}{4.7}{5}
	
	\barra{4}{TTL/CMOS}{0.2}{0.8}{1.5}{2}{4.7}{5}
	
	\barra{8}{TTL}{0.35}{0.8}{1.5}{2}{3.3}{5}
	
	\barra{12}{LVTTL}{0.4}{0.8}{1.5}{2}{2.4}{3.6}	

\end{tikzpicture}
\end{page}


\end{document}