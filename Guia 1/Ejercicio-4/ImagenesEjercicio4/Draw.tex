\documentclass[border={0.5cm 0.5cm 0.5cm 0.5cm}, 11pt, tikz, multi=page]{standalone}
\usepackage[utf8]{inputenc}
\usepackage[spanish, es-tabla, es-noshorthands]{babel}

\usepackage[a4paper, footnotesep = 1cm, width=18cm, left=2cm, top=2.5cm, height=25cm, textwidth=18cm, textheight=25cm]{geometry}
%\geometry{showframe}

\usepackage{tikz}
\usepackage{textcomp}
\usetikzlibrary{shapes,arrows}

\usepackage{amsmath}
\usepackage{amsfonts}
\usepackage{amssymb}
\usepackage{float}
\usepackage{graphicx}
\usepackage{caption}
\usepackage{subcaption}
\usepackage{multicol}
\usepackage{multirow}
\setlength{\doublerulesep}{\arrayrulewidth}
\usepackage{booktabs}
\usepackage{pgfplots}

\usepackage{hyperref}
\hypersetup{
    colorlinks=true,
    linkcolor=blue,
    filecolor=magenta,      
    urlcolor=blue,
    citecolor=blue,    
}

\newcommand{\quotes}[1]{``#1''}
\usepackage{array}
\newcolumntype{C}[1]{>{\centering\let\newline\\\arraybackslash\hspace{0pt}}m{#1}}
\usepackage[american]{circuitikz}
\usepackage{fancyhdr}
\usepackage{units}

% Definition of blocks:
\tikzset{%
  block/.style    = {draw, thick, rectangle, minimum height = 3em,
    minimum width = 3em},
  sum/.style      = {draw, circle, node distance = 2cm}, % Adder
  input/.style    = {coordinate}, % Input
  output/.style   = {coordinate}, % Output
  >=Stealth
}

% Defining string as labels of certain blocks.
\newcommand{\suma}{\Large $\Sigma$}
\newcommand{\inte}{$\displaystyle \int$}
\newcommand{\derv}{\huge $\frac{d}{dt}$}
\usetikzlibrary{patterns}

% defining the new dimensions and parameters
\newlength{\hatchspread}
\newlength{\hatchthickness}
\newlength{\hatchshift}
\newcommand{\hatchcolor}{}
% declaring the keys in tikz
\tikzset{hatchspread/.code={\setlength{\hatchspread}{#1}},
         hatchthickness/.code={\setlength{\hatchthickness}{#1}},
         hatchshift/.code={\setlength{\hatchshift}{#1}},% must be >= 0
         hatchcolor/.code={\renewcommand{\hatchcolor}{#1}}}
% setting the default values
\tikzset{hatchspread=3pt,
         hatchthickness=0.4pt,
         hatchshift=0pt,% must be >= 0
         hatchcolor=black}
% declaring the pattern
\pgfdeclarepatternformonly[\hatchspread,\hatchthickness,\hatchshift,\hatchcolor]% variables
   {custom north west lines}% name
   {\pgfqpoint{\dimexpr-2\hatchthickness}{\dimexpr-2\hatchthickness}}% lower left corner
   {\pgfqpoint{\dimexpr\hatchspread+2\hatchthickness}{\dimexpr\hatchspread+2\hatchthickness}}% upper right corner
   {\pgfqpoint{\dimexpr\hatchspread}{\dimexpr\hatchspread}}% tile size
   {% shape description
    \pgfsetlinewidth{\hatchthickness}
    \pgfpathmoveto{\pgfqpoint{0pt}{\dimexpr\hatchspread+\hatchshift}}
    \pgfpathlineto{\pgfqpoint{\dimexpr\hatchspread+0.15pt+\hatchshift}{-0.15pt}}
    \ifdim \hatchshift > 0pt
      \pgfpathmoveto{\pgfqpoint{0pt}{\hatchshift}}
      \pgfpathlineto{\pgfqpoint{\dimexpr0.15pt+\hatchshift}{-0.15pt}}
    \fi
    \pgfsetstrokecolor{\hatchcolor}
%    \pgfsetdash{{1pt}{1pt}}{0pt}% dashing cannot work correctly in all situation this way
    \pgfusepath{stroke}
   }

\begin{document}

%%%%%%%%%%%%%%%%%%%%%%%%%%%%%%%%%%%%%%%%%%%%%%%%%%%%%%%%%%%%%%%%%%%%%
%							COMANDOS								%
%%%%%%%%%%%%%%%%%%%%%%%%%%%%%%%%%%%%%%%%%%%%%%%%%%%%%%%%%%%%%%%%%%%%%

%Eje vertical que compara los niveles
\newcommand{\barra}[8] % #1 = Xpos #2 = name #3 = VOL #4 = VIL #5 = Vt #6 = VIH #7 = VOH #8 = VCC
{
	\node [below] at (#1,-0.5) {#2};
	
	%Barras anchas
	\draw[-][draw=red, line width=2mm] (#1,#7) -- (#1,#8);
	\draw[-][draw=red, line width=2mm] (#1,0) -- (#1,#3);
	
	\draw[-][draw=gray, line width=2mm] (#1,#6) -- (#1,#7);
	\draw[-][draw=gray, line width=2mm] (#1,#3) -- (#1,#4);	
	
	\draw[-][draw=lightgray, line width=2mm] (#1,#4) -- (#1,#6);	
	
	%Eje vertical limites
	\draw[-][draw=black, very thick] (#1,0) -- (#1,#8);
	\draw[-][draw=black, very thick] (-0.2 + #1,0) -- (0.2 + #1,0);
	\draw[-][draw=black, very thick] (-0.2 + #1,#8) -- (0.2 + #1,#8);
	
	\node [left] at (-0.2 + #1,#8) {#8 V};
	\node [right] at (0.2 + #1,#8) {$V_{CC}$};	
	\node [left] at (-0.2 + #1,0) {0 V};
	\node [right] at (0.2 + #1,0) {$GND$};
	
	%Punto medio
	\draw[-][draw=black, very thick] (-0.1 + #1,#5) -- (0.1 + #1,#5);
	\node [left] at (-0.1 + #1,#5) {#5 V};
	\node [right] at (0.1 + #1,#5) {$V_t$};
	
	%VOH
	\draw[-][draw=black, very thick] (-0.1 + #1,#7) -- (0.1 + #1,#7);
	\node [left, color = gray] at (-0.1 + #1,#7) {#7 V};
	\node [right, color = gray] at (0.1 + #1,#7) {$V_{OH}$};
	
	%VIL
	\draw[-][draw=black, very thick] (-0.1 + #1,#4) -- (0.1 + #1,#4);
	\node [left, color = gray] at (-0.1 + #1,#4) {#4 V};
	\node [right, color = gray] at (0.1 + #1,#4) {$V_{IL}$};
	
	%VIH
	\draw[-][draw=black, very thick] (-0.1 + #1,#6) -- (0.1 + #1,#6);
	\node [left, color = red] at (-0.1 + #1,#6) {#6 V};
	\node [right, color = red] at (0.1 + #1,#6) {$V_{IH}$};
	
	%VOL
	\draw[-][draw=black, very thick] (-0.1 + #1,#3) -- (0.1 + #1,#3);
	\node [left, color = red] at (-0.1 + #1,#3) {#3 V};
	\node [right, color = red] at (0.1 + #1,#3) {$V_{OL}$};	
	
}

%%%%%%%%%%%%%%%%%%%%%%%%%%%%%%%%%%%%%%%%%%%%%%%%%%%%%%%%%%%%%%%%%%%%%
%								DIBUJOS								%
%%%%%%%%%%%%%%%%%%%%%%%%%%%%%%%%%%%%%%%%%%%%%%%%%%%%%%%%%%%%%%%%%%%%%

%DIBUJO 1
\begin{page}
\begin{tikzpicture}[yscale=1]

	%#1 = Xpos #2 = name #3 = VOL #4 = VIL #5 = Vt #6 = VIH #7 = VOH #8 = VCC
	
	\barra{0}{5 V TTL}{0.4}{0.8}{1.55}{2}{2.4}{5}
	
	\barra{5}{3.3 V LVTTL}{0.4}{0.8}{1.55}{2}{2.4}{3}	

\end{tikzpicture}
\end{page}

%DIBUJO 3
\begin{page}
\begin{circuitikz}[american voltages]	
		
	\draw
		(0,0) node[nigfete,solderdot, rotate = -90](fet){}
		
		(fet.S) -| node[pos=.5, circ](r1){} (-2,-1) node[ocirc, label=below:$3.3 \ v$ Signal](){}
		(fet.D) -| node[pos=.5, circ](r2){} (2,-1) node[ocirc, label=below:$5 \ v$ Signal](){}
		
		(r1.center) to[R, label=$10 \ K$] ++ (0,2) node[circ](aux1){} -| (fet.G)
		(aux1.center) -- ++ (-1,0) node[ocirc, label=above:$3.3 \ V$](){}
		
		(r2.center) to[R, label=$10 \ K$] ++ (0,2) -- ++ (1,0) node[ocirc, label=above:$5 \ V$](){}		
	;

\end{circuitikz}
\end{page}


\end{document}