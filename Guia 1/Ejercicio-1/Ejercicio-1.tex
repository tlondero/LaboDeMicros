\input{../Informe/Header.tex}

\begin{document}
\subsection{Implementación por lógica discreta}
Para la lógica combinacional lo primero que se hizo fue escribir las posiciones de memoria en las que vivirá nuestro periférico en binario.
\begin{align}
0x4000 \implies 16384 \implies 0100 0000 0000\\
0x2000 \implies 8192(Memory)\\
0x6000 \implies 24576 \implies 011000000000 
\end{align}
De aqui se arma la tabla de verdad de los últimos 3 bits mas significativos.
\begin{table}[H]
\centering
\begin{tabular}{cccc}
$a_{15}$ & $a_{14}$ & \multicolumn{1}{c|}{$a_{13}$} & CS \\ \hline
0 & 0 & 0 & 0 \\
0 & 0 & 1 & 0 \\
0 & 1 & 0 & 1 \\
0 & 1 & 1 & 1 \\
1 & 0 & 0 & 0 \\
1 & 0 & 1 & 0 \\
1 & 1 & 0 & 0 \\
1 & 1 & 1 & 0
\end{tabular}
\label{tab:truetab}
\end{table}
De aquí se puede solucionar el mapa de karnaugh para la siguiente configuración:
\begin{figure}[H]
  \centering
  \includegraphics[width=0.5\textwidth,page = 1]{ImagenesEjercicio1/Circuits.pdf}
  \caption{Lógica Discreta}.
  \label{fig:circLog}
\end{figure}
\subsection{Implementación por lógica de baja complejidad}
Se utilizó el decodificador 74LS139, conectando a los pines $a_{15}$ y $a_{14}$ a las entradas B y A respectivamente, el CS será la salida $Y_1$ quedando de la siguiente manera
\begin{figure}[H]
  \centering
  \includegraphics[width=0.5\textwidth,page = 2]{ImagenesEjercicio1/Circuits.pdf}
  \caption{Lógica de baja complejidad}.
  \label{fig:circdec}
\end{figure}
\subsection{Implementación por medio de una PAL}
Se utilizó una PAL como decodificador de direcciones,como se observa en la tabla (\ref{tab:truetab}) es posible detectar el perisferico viendo únicamente los bits $a_{15}$ y $a_{14}$ asi se llega a la siguiente ecuación:
\begin{align}
x_1 = a_{15} \ \ \  \  \  \  \  \  \  \ x_2=a_{14} \\
f1=CS \ \ \ f1= \bar{x_1} \  \&  \ x_2
\end{align}

\subsection{Análisis y construcción del diagrama de tiempos}
Se construyó para el microprocesador M68HC11 el diagrama de tiempos para un ciclo de lectura/escritura, usando como ejemplo la posición de memoria \$2345, la cual está dentro de la hipotética región del mapa de memoria donde se aloja la memoria para la cual se diseñó el decodificador anteriormente.

\begin{figure}[H]
  \centering
  \includegraphics[width=\textwidth]{ImagenesEjercicio1/diagtiempos.png}
  \caption{Ciclo de lectura/escritura de \textit{DATA} en la dirección de memoria \textit{\$2345}}.
  \label{diagtiempos}
\end{figure}
 
Para el análisis de tiempos se tiene en cuenta una frecuencia característica de $2 \ MHz$. Dado esto, se obtiene un rise time de las señales de $t_4 = 20 \ ns$ y un periodo entre ciclos de lectura/escritura de $t_1 = 500 \ ns$, por lo que los tiempos en alto y bajo de la señal \textbf{E} de enable serán de $t_3 = 230 ns$ respectivamente. 

\subsubsection{Primera mitad del ciclo de escritura/lectura}

El comienzo del ciclo de lectura o escritura comienza con el flanco descendente de la señal de enable. Un tiempo $t_{26} = 53 \ ns$ después se activa la señal \textbf{AS} de address strobe, lo cual indica que se utilice el bus de address entero para cargar la parte baja y alta de la dirección de memoria en los puertos C y B del M68HC11 respectivamente. Esta señal se desactiva luego de un tiempo $t_{27} = 96 \ ns$ activando el latch que retendrá la parte baja de la dirección de memoria. De esta manera se logra multiplexar la parte baja del bus de address, o puerto C, para leer o escribir datos al igual que retener la parte baja de la dirección del mapa de memoria.

El puerto C tiene la dirección de memoria por un tiempo válido de $t_{t22} = 88 \ ns$ como mínimo y el puerto B por un tiempo de $t_{12} = 94 \ ns$ como mínimo, que corresponde con el flanco ascendente de la señal de enable y marca la mitad del ciclo de lectura/escritura.

\subsubsection{Segunda mitad del ciclo de escritura/lectura} 
\textbf{Lectura:}
En el caso de la lectura, el tiempo de setup para que el periférico coloque el dato a su salida y lo mantenga estable antes del flanco descendente de la señal de enable es de $t_{17} = 30 \ ns$ y debe ser mantenido estable por $t_{18A} = 10 \ ns$ pasado dicho flanco. Luego pasa a hiZ el puerto C pasados $t_{18B} = 83 \ ns$ de dicho flanco.

\textbf{Escritura:}
Para el caso de la escritura, el puerto C tiene un delay máximo para contener el dato a escribir de $t_{19} = 128 \ ns$ y un tiempo de hold de $t_{21} = 33 \ ns$ como mínimo, por lo cual el tiempo de escritura deberá ser como máximo de $t_{3} + t_{21} - t_{19} = 143 \ ns$.

Finalmente, el address se mantendrá por un tiempo de $t_9$ tras el flanco descendente de la señal de enable, por lo que el tiempo válido de lectura de la dirección de memoria en un ciclo de $t_1 = 500ns$ será de $t_1 - t_{26} + t_{9} = 480 \ ns$.
 
 
 
 
 
 
 
 
 
 
 
 
 
 
 
 
 
 
 
 
 
 
 
 
 
 
 
 
 
 
 
 
%\begin{table}[H]
%\centering
%\begin{tabular}{cccc}
%\hline
%\textbf{A15} & \textbf{A14} & \textbf{A13} & \textbf{CS} \\
%\hline
%0            & 0            & 0            & 0           \\
%0            & 0            & 1            & 0           \\
%\textcolor{red}{0}           & \textcolor{red}{1}            %& \textcolor{red}{0}            & \textcolor{red}{1}           %\\
%\textcolor{red}{0}            & \textcolor{red}{1}            %& \textcolor{red}{1}            & \textcolor{red}{1}           %\\
%1            & 0            & 0            & 0           \\
%1            & 0            & 1            & 0           \\
%1            & 1            & 0            & 0           \\
%1            & 1            & 1            & 0          \\
%\hline
%\end{tabular}
%\end{table}

\end{document}