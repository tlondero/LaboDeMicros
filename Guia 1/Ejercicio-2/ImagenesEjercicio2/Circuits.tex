\documentclass[border={0.5cm 0.5cm 0cm 0.5cm}, 11pt, tikz, multi=page]{standalone}
\usepackage[utf8]{inputenc}
\usepackage[spanish, es-tabla, es-noshorthands]{babel}

\usepackage[a4paper, footnotesep = 1cm, width=18cm, left=2cm, top=2.5cm, height=25cm, textwidth=18cm, textheight=25cm]{geometry}
%\geometry{showframe}

\usepackage{tikz}
\usepackage{textcomp}
\usetikzlibrary{shapes,arrows}

\usepackage{amsmath}
\usepackage{amsfonts}
\usepackage{amssymb}
\usepackage{float}
\usepackage{graphicx}
\usepackage{caption}
\usepackage{subcaption}
\usepackage{multicol}
\usepackage{multirow}
\setlength{\doublerulesep}{\arrayrulewidth}
\usepackage{booktabs}
\usepackage{pgfplots}

\usepackage{hyperref}
\hypersetup{
    colorlinks=true,
    linkcolor=blue,
    filecolor=magenta,      
    urlcolor=blue,
    citecolor=blue,    
}

\newcommand{\quotes}[1]{``#1''}
\usepackage{array}
\newcolumntype{C}[1]{>{\centering\let\newline\\\arraybackslash\hspace{0pt}}m{#1}}
\usepackage[american]{circuitikz}
\usepackage{fancyhdr}
\usepackage{units}

% Definition of blocks:
\tikzset{%
  block/.style    = {draw, thick, rectangle, minimum height = 3em,
    minimum width = 3em},
  sum/.style      = {draw, circle, node distance = 2cm}, % Adder
  input/.style    = {coordinate}, % Input
  output/.style   = {coordinate}, % Output
  >=Stealth
}

% Defining string as labels of certain blocks.
\newcommand{\suma}{\Large $\Sigma$}
\newcommand{\inte}{$\displaystyle \int$}
\newcommand{\derv}{\huge $\frac{d}{dt}$}

\begin{document}

%%%%%%%%%%%%%%%%%%%%%%%%%%%%%%%%%%%%%%%%%%%%%%%%%%%%%%%%%%%%%%%%%%%%%
%							COMPONENTES								%
%%%%%%%%%%%%%%%%%%%%%%%%%%%%%%%%%%%%%%%%%%%%%%%%%%%%%%%%%%%%%%%%%%%%%

%Mux
\newcommand{\Mux}[1] % #1 = name
{node(#1_origin){}
	node[dipchip, num pins=8, hide numbers, no topmark, external pins width=0](C3){}
	
	($ (C3.bpin 1) !.5! (C3.bpin 8) $) ++ (0,0.5) node[](){$74LS139$}

	%LEFT NODES
	node[right, font=\footnotesize] at (C3.bpin 1) {$A$}
	node[right, font=\footnotesize] at (C3.bpin 2) {$B$}	
	node[right, font=\footnotesize] at (C3.bpin 4) {$E$}		

	(C3.bpin 1) -- ++ (-0.5,0) node[](#1_a){}	
	(C3.bpin 2) -- ++ (-0.5,0) node[](#1_b){}
	(C3.bpin 4) -- ++ (-0.5,0) node[](#1_e){}
	(C3.bpin 4) -- ++ (-0.075,0) node[ocirc](){}

	%RIGHT NODES
	node[left, font=\footnotesize] at (C3.bpin 5) {$Y_0$}	
	node[left, font=\footnotesize] at (C3.bpin 6) {$Y_1$}	
	node[left, font=\footnotesize] at (C3.bpin 7) {$Y_2$}
	node[left, font=\footnotesize] at (C3.bpin 8) {$Y_3$}
	
	(C3.bpin 5) -- ++ (0.5,0) node[](#1_y0){}
	(C3.bpin 6) -- ++ (0.5,0) node[](#1_y1){}
	(C3.bpin 7) -- ++ (0.5,0) node[](#1_y2){}
	(C3.bpin 8) -- ++ (0.5,0) node[](#1_y3){}	
	
	(C3.bpin 5) -- ++ (0.075,0) node[ocirc](){}
	(C3.bpin 6) -- ++ (0.075,0) node[ocirc](){}
	(C3.bpin 7) -- ++ (0.075,0) node[ocirc](){}
	(C3.bpin 8) -- ++ (0.075,0) node[ocirc](){}
	
}


%%%%%%%%%%%%%%%%%%%%%%%%%%%%%%%%%%%%%%%%%%%%%%%%%%%%%%%%%%%%%%%%%%%%%
%							CIRCUITOS								%
%%%%%%%%%%%%%%%%%%%%%%%%%%%%%%%%%%%%%%%%%%%%%%%%%%%%%%%%%%%%%%%%%%%%%

%2 
\begin{page}
\tikzset{mux/.style={muxdemux, muxdemux def={Lh=2, NL=1, Rh=5, NR=2, NB=0, NT=1, w=2, inset w=1, inset Lh=0, inset Rh=2, square pins=1}}}
\begin{circuitikz}[american voltages]

	\def\xspace{0.75}
		
	\draw
		%DIBUJO LOS PUNTOS a
		(0,0) node[circ, label=above:$a_{15}$](a15){}
		(a15) ++ (\xspace ,0) node[circ, label=above:$a_{14}$](a14){}
		(a14) ++ (\xspace ,0) node[circ, label=above:$a_{13}$](a13){}
		(a13) ++ (\xspace ,0) node[circ, label=above:$\overline{a_{12}}$](a12){}
		(a12) ++ (\xspace ,0) node[circ, label=above:$a_{11}$](a11){}
		(a11) ++ (\xspace ,0) node[circ, label=above:$\overline{a_{10}}$](a10){}
		(a10) ++ (\xspace ,0) node[circ, label=above:$\overline{a_{9}}$](a9){}
		(a9) ++ (\xspace ,0) node[circ, label=above:$\overline{a_{8}}$](a8){}
		(a8) ++ (\xspace ,0) node[circ, label=above:$\overline{a_{7}}$](a7){}
		(a7) ++ (\xspace ,0) node[circ, label=above:$\overline{a_{6}}$](a6){}
		(a6) ++ (\xspace ,0) node[circ, label=above:$\overline{a_{5}}$](a5){}
		(a5) ++ (\xspace ,0) node[circ, label=above:$\overline{a_{4}}$](a4){}
		(a4) ++ (\xspace ,0) node[circ, label=above:$\overline{a_{3}}$](a3){}
		(a3) ++ (\xspace ,0) node[circ, label=above:$\overline{a_{2}}$](a2){}
		(a2) ++ (\xspace ,0) node[circ, label=above:$\overline{a_{1}}$](a1){}
		(a1) ++ (\xspace ,0) node[circ, label=above:$\overline{a_{0}}$](a0){}
		
		
		%CONECTO a CON AND ENORME
		(a0.center) |- node[pos=.5, circ](){} ++ (1,-1.5) node[american and port, number inputs = 15, anchor = in 1, yscale = 1.5](and1){}
		(a1.center) |- node[pos=.5, circ](){} (and1.in 2)
		(a2.center) |- node[pos=.5, circ](){} (and1.in 3)
		(a3.center) |- node[pos=.5, circ](){} (and1.in 4)
		(a4.center) |- node[pos=.5, circ](){} (and1.in 5)
		(a5.center) |- node[pos=.5, circ](){} (and1.in 6)
		(a6.center) |- node[pos=.5, circ](){} (and1.in 7)
		(a7.center) |- node[pos=.5, circ](){} (and1.in 8)
		(a8.center) |- node[pos=.5, circ](){} (and1.in 9)
		(a9.center) |- node[pos=.5, circ](){} (and1.in 10)
		(a10.center) |- node[pos=.5, circ](){} (and1.in 11)
		(a12.center) |- node[pos=.5, circ](){} (and1.in 12)
		(a13.center) |- node[pos=.5, circ](){} (and1.in 13)
		(a14.center) |- node[pos=.5, circ](a14p){} (and1.in 14)
		(a15.center) |- node[pos=.5, circ](a15p){} (and1.in 15)	
		
		(and1.bin 14) ++ (-0.075,0) node[ocirc](){}
		
	
		%CONECTO MUX
		(and1.out) -- ++ (0.5,0) node[mux, xscale = 0.75, yscale = 0.75, anchor = lpin 1](mu){}
		(mu.brpin 1) node[label=left:$1$](){}
		(mu.brpin 2) node[label=left:$0$](){}
		
		(mu.rpin 1) -- ++ (0.5,0) node[circ, label=right:$CS3$](){}
		(mu.rpin 2) -- ++ (0.5,0) node[circ, label=right:$CS2$](){}
		
		(a11.center) |- (mu.tpin 1)
		
		
		%DIBUJO SEGUNDA AND		
		(and1) ++ (0,-3) node[american and port, number inputs = 4, yscale = 1.5](and2){}
		(a12.center) |- node[pos=.5, circ](){} (and2.in 1)
		(a13.center) |- node[pos=.5, circ](){} (and2.in 2)
		(a14p.center) |- node[pos=.5, circ](a14pp){} (and2.in 3)
		(a15p.center) |- node[pos=.5, circ](a15pp){} (and2.in 4)
		
		(and2.out) -- ++ (1,0) node[circ, label=right:$CS1$](){}
		
		(and2.bin 3) ++ (-0.075,0) node[ocirc](){}
		(and2.bin 4) ++ (-0.075,0) node[ocirc](){}
		
		%DIBUJO TERCER AND		
		(and2) ++ (0,-3) node[american and port, yscale = 1.5](and3){}
		(a14pp.center) |- node[pos=.5, circ](){} (and3.in 1)
		(a15pp.center) |- node[pos=.5, circ](){} (and3.in 2)
		
		(and3.out) -- ++ (1,0) node[circ, label=right:$CS0$](){}
		
		
		
		
		

		
		
		
			
		
	;

%	%NODOS AUXILIARES
%	\draw[color = red]	
%	
%	;

\end{circuitikz}
\end{page}





\end{document}