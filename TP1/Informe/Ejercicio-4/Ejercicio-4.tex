\documentclass[a4paper]{article}
\usepackage[utf8]{inputenc}
\usepackage[spanish, es-tabla, es-noshorthands]{babel}
\usepackage[table,xcdraw]{xcolor}
\usepackage[a4paper, footnotesep = 1cm, width=20cm, top=2.5cm, height=25cm, textwidth=18cm, textheight=25cm]{geometry}
%\geometry{showframe}

\usepackage{tikz}
\usepackage{amsmath}
\usepackage{amsfonts}
\usepackage{amssymb}
\usepackage{float}
\usepackage{graphicx}
\usepackage{caption}
\usepackage{subcaption}
\usepackage{multicol}
\usepackage{multirow}
\setlength{\doublerulesep}{\arrayrulewidth}
\usepackage{booktabs}

\usepackage{hyperref}
\hypersetup{
    colorlinks=true,
    linkcolor=blue,
    filecolor=magenta,      
    urlcolor=blue,
    citecolor=blue,    
}

\newcommand{\quotes}[1]{``#1''}
\usepackage{array}
\newcolumntype{C}[1]{>{\centering\let\newline\\\arraybackslash\hspace{0pt}}m{#1}}
\usepackage[american]{circuitikz}
\usetikzlibrary{calc}
\usepackage{fancyhdr}
\usepackage{units} 
\usepackage{svg}

\graphicspath{{../Ejercicio-1/}{../Ejercicio-2/}{../Ejercicio-3/}{../Ejercicio-4/}{../Ejercicio-5/}}
%\svgpath{{../Ejercicio-1/}{../Ejercicio-2/}{../Ejercicio-3/}{../Ejercicio-4/}{../Ejercicio-5/}}

\pagestyle{fancy}
\fancyhf{}
\lhead{22.05 ASSD}
\rhead{Mechoulam, Lambertucci, Rodriguez, Londero}
\rfoot{Página \thepage}

\begin{document}

\begin{center}
\textcolor{black}{\LARGE{\textbf{\underline{OPEN DOOP: Manual de usuario}}}}
\end{center}


\multicols{2}{

\section*{Inicio:}

Al encender el dispositivo, se esperará que se ingrese un ID, utilizando el encoder ó la tarjeta. No ocurre nada en caso de que la ID introducida no sea correcta. Una vez ingresada la ID, se encenderá el led \textcolor{blue}{azul} y deberá ingresar el PIN. Con el botón del \textbf{SW2} se puede cancelar la introducción del PIN/ID, mientras que con el \textbf{SW3} se borra el dígito introducido, volviendo al anterior.

Se cuenta con 3 intentos para ingresar dicho pin. En el caso de que el introducido no sea correcto luego de los intentos, se bloqueará dicho ID, hasta que un \textbf{ADMIN} lo desbloquee.

\section*{Acces:}

Una vez dentro es posible navegar a través de las distintas opciones mediante el uso del encoder. Las opciones disponibles son ``OPEN'', ``USERS'' y ``BRIGHT''. Open permite abrir la puerta mientras que las otras dos opciones permiten configurar el sistema.

En el caso de inactividad, es decir, que no se efectúe ninguna acción, se volverá al menú principal automáticamente, deslogueando al usuario.

\section*{Menúes:}

\subsection*{Brightness:}

Permite configurar el nivel de brillo del display, existiendo 3 distintos niveles.

\subsection*{Users:}

En este puede configurarse 3 opciones distintas, de las cuales dos son exclusivas para administradores.

\subsubsection*{Clave:}

En este submenu, disponible para cualquier usuario, permite cambiar el PIN de seguridad del usuario.

\subsubsection*{Delete:}

\textbf{SOLO PARA ADMINS.} En este submenu permite eliminar usuarios existentes en la base de datos.

\subsubsection*{Add:}

\textbf{SOLO PARA ADMINS.} En este submenu permite agregar usuarios nuevos a la base de datos.

}

\end{document}